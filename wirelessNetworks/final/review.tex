\documentclass{article}
\title{Introduction to Wireless Networks Final Review}
\begin{document}
\section{Multiplexing}
We want to allow multiple users to access the same channel. We usually split the
available spectrum. We also have the frequency and time domains to be wary of. 

\textit{frequency domain}: The frequency domain refers to the measuring of a 
mathematical function with respect to its frequency rather than with respect to time.

\textit{time domain}: measuring of functions with respect to time, rather than frequency. 

When we have mutliple users that want to share a channel, we need to think of some way to 
split that channel, this is called \textbf{multiplexing}.

There are usually two major ways of splitting the channel, frequency division and time division.

With frequency division, we assign each comepting signal a special frequency to transmit.

With time division, we split a \textit{frame} into multiple \textit{slots}, which can then be 
occupied by the signals that wish to transmit instructions. Both the uplink and downlink channels
will employ this. However, from the diagram in the ppt, it seems that the slot link in the 
uplink will occupy the nth slot in the downlink. The ith packet in the uplink will occupy the 
n-ith slot in the downlink.

\subsection{Code Division Multiple Access}
With this scheme, we can  use a ``spread spectrum" technique to generate a code for each individual
sender. We break each bit according to a code (usually with bitwise XOR) to generate a code. Then,
we can simultaneously send data for multiple users at once. 

This introduces the \textbf{Near-Far problem}, which denotes the phenomenon where a signal that is
closer to the receiver will be heard stronger and even potentially block one that is farther away. There
are a couple kinds of interference in CDMA, such as \textit{adjacent channel interference}.

\subsection{Narrowband systems}
There are a large number of narrowband systems, such as:
\begin{enumerate}
		\item{\textbf{FDD}}
		\item{\textbf{Narrrowband TDMA and Narrowband FDMA}}
		\item{\textbf{FDMA\ FDD}}
		\item{\textbf{FDMA\ TDD}}
		\item{\textbf{TDMA\ TDD}}
\end{enumerate}

\subsection{Frequency Division Duplexing}
For FDD, we have two different bands, the forward and reverse band. We need a duplexer here to The 
frequency separating the reverse and forward channel is always constant.
\subsection{Time division Multiplexing}
Using this scheme, we use time for forward and reverse link. Then multiple users can share a single
radio channel. There is a different reverse and forward time slot.

\subsection{wideband Systems}
In this scheme there are a large nubmer of transmitters on one channel. 
\subsection{Frequency Dvision Multiplexing Access}
In this scheme, there is one circuit per channel. Because there is only one circuit (circuit switching)
per channel, if the channel is not in use, then it will not be as efficient(similar to circuit switching
in networks). This is usually implemented in narrowband systems.
\end{document}

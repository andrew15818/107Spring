\documentclass{article}
\usepackage[margin=0.5in]{geometry}
\title{Data Structures}
\begin{document}
\maketitle
\section{Elementary Data Structures}
\subsection{Stacks and Queues}
Stacks and queues are \textit{dynamic} data structures in which we are constantly modifying the 
elements of a partiular set. In a stack, if we want to remove an item, we ususally remove the 
last element in the stack, and we call it a \textit{last-in-first-out} type of  data structure, 
since the last element to ocme in is the furst element we remove.
\subsubsection{Stacks}
As mentioned, stacks are \textit{first-in-first-out} data structures, which means that only the 'top' element
is accessible at any given time. The operations we may perform are all in O(1), meaning that they are constant
time operations. These include:
\begin{enumerate}
		\item{\texttt{Push(S,value)}}: When we push a value onto the stack, we are adding an element to the array 
				and moving the \texttt{S.top} attribute, increasing it by 1.
		\item{\texttt{Pop()}}: When we pop an item, there is no need for an argument. We just reset \texttt{S.top}
				down by 1, and potentially overwrite the data item when we have to.
\end{enumerate}
Essentially, a stack behaves like an array with n element, and we can only use the topmost one. If the top value
exceeds n, we say that \textbf{stack overflow} (ayyy lmao)has occurred. The opposite, when  we try to pop an 
empty stack, we say that \textbf{stack underflow} has ocurred.
\subsubsection{Queues}
For a queue, we can perform similar operations to a stack, except that they have different names. For example, an insert
operation is called \texttt{enqueue}, and a delete operation is called a \texttt{dequeue}.

We can compare a queue to a line of customers, where the first one in line is the first one to proceed. The elements all
come into the back of the queue, and elements are only taken out of the front, or \textit{head}. 

Another distinctive feature of queues is that the elements 'wrap around', that is, if the queue goes to the end of the 
array, the next element will be arr[0]. When the tail = head -1, the queue is full, since it wrapped around to the other
side and will interfere if it grows any more. This is because we are constantly removing elements from the 
head of the queue, so if we didn't implement this, our queue would be decreasing in size every time 
we dequeue.

For queues, because of its FIFO property, whenever we \texttt{dequeue}, the first element in the queue is
removed, similar to the first customer in a line that proceeds to the counter. Similarly, the element that 
is \texttt{enqueued} always goes at the tail of the queue. 
\subsection{10.2 Linked Lists}
A \textbf{\textit{linked list}} is a container-type data structure in which the elements are categorized not
as indices in an array, rather each element, or \textit{node}, has a pointer to the next object. This means
that linked lists don't need to be of a specified size, they can expand and contract dynamically.

This would be an example of a singly(?) linked list. A \textit{\textbf{doubly linked list}}, however, contains
a pointer to the next and previous element in the list respectively. At first, the head and the tail both 
point to the same element. The first node, n1, should be such that \texttt{n1.prev =null}. This would
let us know that a node in the list is a header. For a tail node, we would only need to chech its \textit{next}
pointer, since the tail pointer should be null as well.

For an unsorted list, if we want to search for a specific element, we would just have to perform a linear search
for our desired element.

Using a \textit{sentinel} object might help us simplify boundary condition checking. We can place another
object \textit{before} the head object and we then have \texttt{nil.next} point to the head object. the object
\texttt{tail.next} could also point to the NIL object. The main reason to even use sentinels is to 
simplify the code maybe within a loop, becuase they don't do that much for reducing complexity.
Also, they take up extra space, so if we are using linked lists with objects and have several of them, it might
be better to not use sentinels.
\subsection{10.3 Implementing Pointers and Objects}
This section covers mostly the pointers implementation in objects that do not support them. Suppose we wanted
to implement a linked list in this way. We could do it with three arrays: one for the key, one for the next and prev. For the next and prev arrays, we have the vaue in the array element i be the index that it corresponds to 
in the key array. For example, the ith value in the prev array is the index of the previous element, so if
prev[i] =5, that objects prev is the one in the 5th index. However, it seems that this approach is very 
limited?
\subsection{Representing Rooted Trees}
Trees can be thought of as an extension of the idea of a linked list. We can have 3 attributes per node. We 
need a pointer to the parent of the node; if \texttt{x.parent} = NIL, then this is the root of the tree.
Considering only binary trees, they can only have two children.

However, we can still implement trees with an arbitrary number of children. We use the \textit{left-child
right-sibling representation}. This combines the idea of a tree and linked list. What this means is that
we have a pointer to the leftmost child. Then, we have a pointer to \texttt{x.right}, which is a pointer
to the right sibling. Then, we have the leftmost pointer serving as the head of the list and pointing
to every right sibling.
\section{Hash Tables}
A \textit{\textbf{hash table}} is a data structure that mimics a 'dictionary', or one in which a \textit{key}
maps to an array index. However, unlike an array where we provide the index directly, here we 
\textit{calculate} the corresponding array index based on a key value. Depending on the function we use,
there might be more or less collisions, or what happens when two different keys map to the same array
index.
\subsection{Direct-address tables}
We can use an array, called a \textit{\textbf{direct-address table}}, to keep track of the data we want to 
associate with the key k. Each element in the array maps to a certain key.
\subsection{Hash Tables}
With direct address tables, we were calculating the value in a table with |U| values, one for every element 
in the universe of possible keys, while really only using |K| values. Instead, we can get away with using a 
much smaller array, and calculating the index necessray for an element. The \textbf{hash function} maps the
entire universe into an array of size n. Since |U|>|M|, there has to be at least one collision (pigeonhole
principle anyone?). One way we can deal is by having a pointer in collisioned indexes to a list of 
elements that map to that same index. This is called \textbf{\textit{chaining}}.

When one or more values map to the index given by h(k), we create a linked list at that index; or rather, we 
have a pointer to the head of a linked list. Then, if we make it a doubly-linked lsit, deletion can run
more efficiently. For search , we could just do a linear search starting at the head, if we just insert 
an item at the end every time. 

Supposing we have a hash table T with a n elements currently being stored and size m, the 
\textit{\textbf{load factor}} $\alpha$ is the ratio n/m (how 'full' the table currently is ?). The load 
factor can vary between 0-1. In reality however, the performance of our hash table depends on the 
function we use to hash the values. 

$n_{j}$ can be used to describe the length of the list at index h(k). The expected length of the list is 
E[$n_{j}$] = $\alpha$.

\textbf{\textit{Theorem 11.1}}

In a hash table where collisions are resolved by sorting, a search takes on average $\Theta$($\alpha$ +1).

\textbf{\textit{Proof}} The time it takes to search for an index k, if there is nothing in the index, is constant.
If there is a list in the index, the expected length of the list is $\alpha$, therefore the worst-case
time for searching the list is O(n). This is all assuming that every element in the array is equally likely 
to be chosen given the original key.
\subsection{Hash Functions}
A good hash function obviously minimizes the probability that two keys map to the same index. We encourage 
simple uniform hashing, since the hash value calculated for every index should be independent of the key 
itself. One popular way is the \textit{division method}, in which we take a (unrelated) prime number, and 
calculate the index by dividing the index and the prime number, then taking the floor of the number. 

Representing the keys as natural numbers would allow us to create functions that operate on them in some way.
For example, translating the number into a certain radix(base). 
\subsection{Division Method}
The simplest way of calculating the 
\end{document}

\documentclass{article}
\usepackage[margin=0.5in]{geometry}
\title{Computer Organization Notes}
\begin{document}
\section{3.6 Parallelism and Computer Arithmetic: Subword Parallelism}
As transistors became more available and accessible, we naturally became more interested in developing 
new functionality for computers. The ability to process other types of data such as video, graphics and
audio, we would need more than 32 bits to interpret. With the new instruction sets, we can operate on
different quantities of bits within a word. For example, traditionally colors and graphics  was represented
with 8 bits. Then, combined with sound options, which we can represent with 16 bits.

For example, the new ARM architecture allows 256 bytes worth of new registers. We can view them as 8 byte
registers, in which case we have 32, or 16 16-bit registers. This ability to perform multiple operations at the 
same time is called \textbf{subword parallelism}.
\section{3.9 Fallacies and Pitfalls}
\subsection{Since shifting left multiplies by two, shifting right divided by two}

This is not always the case. When we shift in values to the most significant bits, we usually will shift 0s.
This might happen when the sign bit is negative, i.e. when we are shifting in a positive sign bit for a number that
is negative. However, even if we try to extend the sing bit, it might lead us to smaller errors but errors nonetheless.
\subsection{Floating-point addition is not associative}
Floating-point additition is not associative, especially when we add two large numbers with a small one. Because the
scientific might not change when we add a small number, if we add a small number it might not make much of a difference.
\subsection{Parallel execution strategies that work for integer data types also work for floating-point data types}
Because integer addition is associative and floating-point addition is not associative. Thus when we rewrite a 
program to run in parallel instead of sequentially, the answer might not be the same.

Thus prgrams even with the same inputs might not produce the same results as code that was run in parallel.
\subsection{THe MIPS instruction add immediate unsigned \texttt{addiu} sign-extends the 16-bit immediate field}
Because this instruction is used for adding unsigned integers,  it really is not that necessary to sign-extend the
number.
\subsection{Only  theoretical mathematicians care about floating-point accuracy}
The Intel Pentium in 1994 had a buf where some of the bits in floating point arithmetic would be wrong occasionally.
What would happen is that floating point division would sometimes return a wrong bits. This was due to incorrect
lookups in the programmable logic array. Thus, values that were supposed to return a +2 would return a 0. In using 
the SRT algorithm for division, it would use a lookup table based on the dividend and divisor. This badly programmed
lookup table is what caused the incorrect values to be retrieved. Only a few of the total entries were even faulty,
5/1066 entries.
\textbf{The concept of the stored program indicates that all bit patterns are essentially meaningless, and can mean 
many different things depending on how they are interpreted.}
\newpage
\section{Chapter 4: The Processor}
\subsection{4.1 Introduction}
All the different types of commands that we have looked at do essentially the same things. We first have to fetch
the instruction from memory and load it. Then, we have to decode the instruction and perform the corresponding
operation. Then, for all instructions (exept jump), we have to use the main ALU. Depending on the opcode and
function code, we can trigger a specific operation to occur. If the command is not a branch or jump instruction,
we incremement the PC by 4 to go to the next instruction.  In our diagram, we also have a \textit{control unit},
whcich determines the output to the control lines of various parts of our CPU, such as multiplexors and ALU.

The control unit takes the instruction as input.
\section{4.2 Logic Design Conventions}
To review: when we are talking about circuits, we can talk about either \textit{combinational} and 
\textit{sequential} circuits. The difference is that the \textit{combinational} circuits depend
only on the current input, whereas \textit{state} circuits or sequential circuits ahve some sort of memory 
component and an use it as an input to the next instruction.

An example of one of these circuits is a \textit{D-flip-flop}, where the inputs are just the value to be written
and the clock. Why do we have the clock as input? It is essentailly because we only want these operations to 
happen when the clock becomes positive. The output at any given time is the previous value.
\subsection{CLocking Methodology}
A clocking methodology refers to the period of time during an operation where we can read and write. If we 
could read and write data at the same time, then the output might be unpredictable.

Thus an \textbf{\textit{edge-triggered methodology}} would allow us to write a value to a register, send it 
through a combinational circuit, and receive the output all in the same cycle. The inputs for the logic 
elements are mostly 32-bits, since that is the size of inputs that our processor accepts.
\section{4.3 Building a Datapath}
The way to determine the next instruction is by using the PC, and use an ALU(which only performs add) to 
find the next instruction. Then, we need to select which operation with a multiplexer, because if it is 
a compare insrtuction, we mgiht have to select that, else we just increment the PC by 4.

For R-type instructions, we have a \textit{register file}, a state element which contains the 32 general 
purpose registers. We then just specify the number of the register to operate on. We will need an ALU to 
work on it though. The R-type instructions require that we work on 3 registers (read 2 and write to 1). We
then need to pass them all as inputs to the register file. The values of the registers that were read are 
then the outputs of the register file. These values are then presumably passed to the main ALU to be operated on.

Since the write control signal is edge triggered, will it be triggered automatically by the main control unit?
It would make sense, since all R-type instructions require a write to happen at the end of the instruction.


For a load instruction of the type \texttt{lw \$t1, offset\_value(\$t2) }, we need to add the signed field of the
offset value to the base register, \$t2 in this case, to the offset field in the intruction. For loads, we need
to then move the instruction into the register specified in the first instruction. When we move some data into
a register, we then have to sign extend, or copy the most significant bit in the field all the way to the end
of the register.

Then we move on to \texttt{beq}. Again, these instructions have three arguments. There are the two registers
that need to be compared for equality. The offset in the field is then added (or subtraced, since it is signed) 
to the address of the current branch instruction. This new address to jump to is called the \textit{branch 
target address}.
 
We need to remember that since we are using the PC as the base address, we are actually using the current PC+4
in our calculations, since we perform that addition in the fetch part of the cycle. Using the ALU, we need 
to check for the equality of the two registers.
\subsection{ Creating a Single Datapath}
Essentially to combine all the different segments we have needed, we use \textit{multiplexors} to select only 
one output depending on a certain input. We would use multiplexors to select just one input for the ALU,
for example to implement the R-type instructions. More specifically, whenever we had to select from 
multiple inputs, such as memory operands and values decoded from the instruction, we would use a multiplexor
to feed the input to the ALU.
\section{4.4 A Simple Implementation Scheme}
How do we select the values to be used by the ALU? It depends on the function field of the instruction. Because
we need to determien the operation to be performed by the ALU, we would need a separate \textit{control unit}.
The job of the control unit is to select one operation to be performed by the ALU. We use 2 bits for the ALUOP,
which we use to tell the control unit if we want to perform a branch, R-type or load/store instruction.

Based on that info, we then (in the case of R-Type operations add,sub,and, or,slt) would need the remaining
6-bit \textit{funct} field found in the instruction. We map the combination of bits to the ALU control input, 
which will go directly to the register. To turn the inputs into gates, we can construct a truth table for the 
inputs. Combining the opcode and funct fields give us a total of 
\begin{equation}
		2^{8} 
\end{equation}
possible inputs. However, most of these inputs are implemented as \textit{don't care} conditions, so we don't 
take them into consideration.

When parsing or decoding the opcode, we fortunately look at the same format regardless of instruciton. The first
4 bits of all different instruction types are the opcode, so we do not really have to imiplement more hardware
becuase \textit{simplicity favors regularity} :).

When we implement the control unit, we will have to impement multiplexors wheneve rwe identify that we need to 
select one input out of many. For example, if we know that there are multiple possible inputs in the 

Let's think about how different instruction types are implemented, or what steps they go through:

1. R-type instructions
\begin{enumerate}
		\item{\textit{load value}}: First, we would load the instruction that is specified by the PC.
		\item{\textit{decode, increse PC}}: Next, we pass the rs,rd,and rt values to the register file.
				Meanwhile, we add 4 to the PC and make that the address of our next isntruction. We also 
				check the opcode and funct values and pass that as ALUsrc.
		\item{\textit{carry out the calculation}}: Based on that info, we already have the proper ALU control 
				input, so we pass the rs and rt to the main ALU, and with the ALU control input, carry out
				the desired operation.
		\item{\textit{store value in register}}: Since we are not writing to memory, the write enable will be 
				disabled, and we skip the memory area entirely, and write the result to the specified register
				in the register file.
\end{enumerate}
 
 \textit{REMEMER}: The wire ALUSrc determines if we will use the the value in the register as the second input 
 or if we will use the address and preform some arithmetic on it. This is independent from the ALUOp, which
 feeds the conrol unit and lets it know what to do.

 For a load instruction,which again is of the form \texttt{lw \$t1, offset(\$ t2)} , we can think of the order of execution as :
 \begin{enumerate}
		 \item{\textit{Fetch instruction, increment PC}}: Same as before, we don't need to use the memory address
				 as the next PC, so we can directly increment by 4.
		 \item{\textit{Add offset value to base address}}: Since in the instruction we specify that the value
				 to be loaded as an offset in the register, we add them in the ALU. However, we first need
				 to shift the base address by 2 and sign-extend the rest. This is the final value we add to
				 the base address.
		 \item{\textit{Write value to register}}: When we finally add the offset to the base address, we need
				 to write the value to the register in the register file, since we would like to use it
				 in later calculations.
 \end{enumerate}
Lastly, for the \texttt{beq} isntruction, which is of the form \texttt{beq \$t1, \$t2, offset}, we perform 
the instructions in a simiar manner to R-Type instructions except with the result of the ALU operation.
\begin{enumerate}
		\item{\textit{Fetch the instruction, increment PC}}: We always fetch the instruction and increment the
				PC, but for branch instructions we need to determine which addresses we will use.
		\item{\item{Subtract the two numbers}}: Again with the sign=extended sign bits, we check the difference
				of the two numbers. If they are equal, the \texttt{ZERO} otuput wire will be asserted. If the 
				branch wire is also asserted from the Control unit, then we will use the target address
				specified in the instruction as our new PC address.
\end{enumerate}
For the beq instruction, we don't use any elements from the data memory part of the unit.
\end{document}

